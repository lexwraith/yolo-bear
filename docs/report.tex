\def \lang {Yolo-Bear }

\documentclass[12pt]{book}
\title{\lang\\A language for rapid resume building and distribution}
\author{Columbia University Students}
\begin{document}
\frontmatter
\maketitle
\chapter{Introduction}

\lang takes in a \lang code file that requires information about a user, with optional parameters for a job description(s). 
It then does the following:
\begin{description}
\item[Raw LaTeX Generation] - Creates a basic LaTeX document with core information and basic formatting
\item[Code Glance] - Looks at user preferences and parameters set on how we want to do things, maybe even pulls job descriptions via URL request. Looks at constraints we set, e.g. "This statement never changes" so that it doesn't do the next step.
\item[Profile Breakdown] - Breaks down the current user data and creates preloaded syntax trees ready to generate. For example, rather than "Directed XXY project to completion", it will curry "[Verb synonymous to Directed] [Noun that is interchangeable with XXY project, if any] to [Synonyms for completion]" as a preloaded statement ready to launch based off of the input job description.
\item[Job Description Analysis] - Uses some library to parse through and look for verbs and nouns and specific things that are connected to the user profile. 
\item[Generation] - Generates a resume based off of as close matching as possible with job description.
\item[Cleanup] - Adds in last portions, e.g. skills/hobbies/whatever, does trimming, sizing, etc to fit on one page or via user description.
\item[Output] - Saves the LaTeX somewhere, and pdflatex's it right away to generate a PDF.

\end{description}
	\section{Background}
		We got tired of perusing through a hundred job descriptions and having to swap out verbs and power statements in our resumes to "tailor" our resume to a job description. It's all buzzwords anyways.
	\section{Related Work}
		BibTex is our inspiration on how we're going to load "profiles". But rather than a bibliography akin to XML/JSON, we're taking in user data.
		Some rough ideas for data type are:
		\begin{description}
			\item[Core] First/Last, Permanent Address, Contact, etc. These are required to even generate a resume.
			\item[Primitives] Verbs (literally verbs, physically are strings), Numbers (Capable of being displayed in percent form), Nouns (People, places, jobs, projects, whatever), Dates (Any format, is convertible).
			\item[Aggregates] Statements - Stmt | Noun Verb Noun [Date] | Noun Verb Stmt \emph{Rough CFG}
		\end{description}
	\section{Goals of \lang}
		\begin{description}
			\item[Adaptive Processes] A \lang programmer should be able to quickly generate tailored resumes to many job descriptions without extensive programming.
			\item[Ease of use] Ultimately, the goal is to allow a \lang programmer to create tailored resumes quickly.
		\end{description}

\tableofcontents
\mainmatter
\chapter{Language Tutorial}
A short explanation telling a novice how to use your language.

\chapter{Language Manual}
Include your language reference manual.

\chapter{Project Plan}
	\section{Planning, Specification, Development, and Testing Process}
		Identify process used for planning, specification, development and testing
	\section{\lang Style Guide}
		Include a one-page programming style guide used by the team
	\section{Project Timeline}
		Show your project timeline
	\section{Team Roles and Responsibilities}
		Identify roles and responsibilities of each team member
	\section{Software Development Environment}
		Describe the software development environment used (tools and languages)
	\section{Project Log}
		Include your project log


\chapter{Architectural Design}
	\section{Block Diagram}
		Give block diagram showing the major components of your translator
	\section{Component Interfaces}
		Describe the interfaces between the components
	\section{Component Credits}
		State who implemented each component

\chapter{Test Plan}
	\section{Test Examples}
		Show two or three representative source language programs along with the target language program generated for each
	\section{Test Suite}
		Show the test suites used to test your translator
	\section{Test Case Rationales}
		Explain why and how these test cases were chosen
	\section{Testing Automation}
		What kind of automation was used in testing
	\section{Testing Credits}
		State who did what

\chapter{Lessons Learned}
	Each team member should explain his or her most important learning
	Include any advice the team has for future teams
	\section{Person 1 Lessons}
	\section{Person 2 Lessons}
	\section{Person 3 Lessons}
	\section{Person 4 Lessons}
	\section{Advice for Future Teams}
\backmatter
\chapter{Appendix}
	Attach a complete code listing of your translator with each module signed by its author
	Do not include any ANTLR-generated files, only the .g sources.
	\section{Code Listing}
\end{document}

