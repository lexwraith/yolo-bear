\documentclass[a4paper]{article}

\usepackage[english]{babel}
\usepackage[utf8]{inputenc}
\usepackage{listings}
\usepackage{color}
\usepackage{courier}
\usepackage{multicol}
\usepackage{float}
\restylefloat{table}

\definecolor{mygray}{rgb}{0.94,0.94,0.94}

\lstset{ %
	backgroundcolor=\color{mygray},
    basicstyle=\footnotesize}


\def \authors{Mikhail Klimentov, Michael Nguyen, Prateek Sinha, Yuchen Zeng, and Eli Bogom-Shanon }

\begin{document}

\textbf{\huge{Language Reference Manual}}

\section{Introduction}

This language reference manual describes the Civ language, developed by \authors for Stephen Edwards's Programming Languages and Translators class (W4115). 

For the most part, this document follows an organizational precedent set by Brian Kernighan and Dennis Ritchie in their "The C Programming Language." 

\section{Lexical Conventions}

\subsection{Comments}

Comments are styled after the C multiline comments. Open with /* and close with */.


{\fontfamily{pcr}\selectfont
\begin{lstlisting} 

   /* This is a comment in Civ */
  
   /*
      This is a multiline comment in Civ
   */
    
   // ILLEGAL: This is NOT a comment in Civ
\end{lstlisting}
}

\subsection{Identifiers}

In Civ, an identifier is an alphanumeric string used for any variables, functions, data definitions, etc. Identifiers cannot begin with symbols (ex: \_toast would not be a valid identifier). Upper case and lower case letters are distinct in Civ. It is advised that identifiers be written in mixedCase notation.

\subsection{Keywords}

The following identifiers are reserved for the use as keywords, and may not be used otherwise:


\begin{multicols}{2}
\noindent 
\center{
{\fontfamily{pcr}\selectfont
boolean \newline
do \newline
else \newline
float \newline
for \newline
if \newline
master \newline
print \newline
return \newline
slave \newline
string \newline
while \newline
void \newline }}
\end{multicols}


\subsection{Constants}

There are no constants in Civ—everything is mutable. 

\subsection{Literals}

Lorem Ipsum

\subsection{Punctuation}

Civ supports standard C punctuation with the exception of the usage of * as a pointer indicator.

\begin{description}
	\item{$[]$} - Brackets are used as indices for lists and list declarations.
	\item{$()$} - Parantheses are used in function calls to surround the function arguments.
    \item{$\{\}$} - Curly braces are used to indicate the beginning end of a body or block statement.
    \item{$,$} - Commas are used in lists generation and as a separator between inline statements.
    \item{$:$} - Colons indicate the beginning of a declaration, and is usually followed by an opening curly brace.
    \item{$;$} - Semicolons tell the compiler that the statement, expression, or body/block is complete.
\end{description}

\subsection{Operators}

\begin{table}[H]
\centering
\begin{tabular}{|c|l|l|}
\hline
Operator & Use & Associativity\\
\hline
+ & Addition & Left\\
\hline
- & Subtraction & Left \\
\hline
* & Multiplication & Left\\
\hline
/ & Division & Left \\
\hline
\% & Modulus & Left \\
\hline
= & Assignment & Right \\
\hline 
== & Equal to & Non-associative\\
\hline
!= & Not equal to & Non-associative\\
\hline 
\textless & Less than & Non-associative\\
\hline 
\textgreater & Less than & Non-associative\\
\hline 
\textless= & Less than or equal to & Non-associative\\
\hline 
\textgreater= & Less than or equal to & Non-associative\\
\hline 
! & Not & Right\\
\hline
\&\& & And & Non-associative\\
\hline
$||$ & Or & Non-associative\\
\hline
\end{tabular}
\caption{Operators}
\end{table}

The precedence of operations is shown as below, from greatest to least precedence: 
\begin{center}

\fontfamily{pcr}\selectfont
* / \% \\
+ - \\ 
! \&\& $||$ \\
\textless \quad  \textgreater \quad \textless= \quad \textgreater= \\
== \quad !=\\
=\\ 
\end{center}


\section{Types}

\subsection{Boolean}
In Civ, a boolean is defined by the values {\fontfamily{pcr}\selectfont true} and {\fontfamily{pcr}\selectfont false}. 

{\fontfamily{pcr}\selectfont
\begin{lstlisting} 
    
    if(true){print("Test")}
    Test
        
    if(false){fork();};
    /*No response*/

\end{lstlisting}
}

\subsection{Float}
All numbers in Civ are floats by default. In the event that the float has no significant digits past the decimal point, it is displayed as an integer, e.g.,

{\fontfamily{pcr}\selectfont
\begin{lstlisting} 

    print(5 + 5);
    10
    
    print(5.0 + 5.1);
    10.1
    
    print(5 + 5.1);
    10.1
    
    print(4.5 + 4.5);
    9

\end{lstlisting}
}

\subsection{String}
Strings are a primitive data $and$ a data structure in Civ, as opposed to a data structure composed of primitive chars. Consequently, String types come with a slew of implicit functions.

{\fontfamily{pcr}\selectfont
\begin{lstlisting}

    String a = "";
    String b = "f";
    String c = "abcdef";
    String d = "1a2b3c";
    
    print(a);
    
    print(b);
    f
    
    print(c);
    abcdef
    
    print(d[2]);
    2
    
    print(a.length);
    0
    
    print(b.length);
    1
    
    /*Note that because everything is mutable,
    d is rebound when its function is called.*/
    print(d.append("test"));
    a12b3ctest
	
\end{lstlisting}
}


\end{document}
