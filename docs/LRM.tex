\documentclass[a4paper]{article}

\usepackage[english]{babel}
\usepackage[utf8]{inputenc}
\usepackage{amsmath}
\usepackage{graphicx}
\usepackage[colorinlistoftodos]{todonotes}

\def \authors{Mikhail Klimentov, Michael Nguyen, Prateek Sinha, Yuchen Zeng, and Eli Bogom-Shanon }

\begin{document}

\textbf{\huge{Language Reference Manual}}

\section{Introduction}

This language reference manual describes the Civ language, developed by \authors for Stephen Edwards's Programming Languages and Translators class (W4115). 

For the most part, this document follows an organizational precedent set by Brian Kernighan and Dennis Ritchie in their "The C Programming Language." 

\section{Lexical Conventions}



\subsection{Tokens}

Lorem Ipsum Dolorum.

\subsection{Comments}

Comments are styled after the C multiline comments. Open with /* and close with */. 

\subsection{Identifiers}

In Civ, an identifier is an alphanumeric string used for any variables, functions, data definitions, etc. Identifiers cannot begin with symbols (ex: \_toast would not be a valid identifier). Upper case and lower case letters are distinct in Civ. It is advised that identifiers be written in mixedCase notation.

\subsection{Constants}

Civ doesn't have constants—everything is mutable.

\subsection{How to Make Sections and Subsections}


\subsection{How to Make Lists}

You can make lists with automatic numbering \dots

\begin{enumerate}
\item Like this,
\item and like this.
\end{enumerate}
\dots or bullet points \dots
\begin{itemize}
\item Like this,
\item and like this.
\end{itemize}
\dots or with words and descriptions \dots
\begin{description}
\item[Word] Definition
\item[Concept] Explanation
\item[Idea] Text
\end{description}

We hope you find write\LaTeX\ useful, and please let us know if you have any feedback using the help menu above.

\end{document}
